\section{The hybrid model}
\label{sec:ImplementationHybrid}

There are a number of aspects that need to be added when considering the hybrid model. These are relations that help connect the electricity and the policy process model.

%%%%%%%%%%%%%%%%
\subsection{KPI calculations}
\label{ssec:elecInvestments}

The KPIs are calculated within the electricity model but they are only used for the hybrid model. There are five secondary indicators that are calculated and two policy core indicators. The secondary indicators are calculated directly from the data that is obtained from the model while the policy core indicators are calculated based on the secondary indicators. Each of the indicators are also normalised as they are to be used within the belief system of the actors within the policy process model. All indicators are calculated for data that is obtained for the year prior to the policy making round. The data considered does not include all of the years between negotiating rounds.

The secondary indicators are the following: renewable energy production (S1), electricity prices (S2), renewable energy investment level (S3), domestic level emissions (S4) and imported emissions (S5).

The renewable energy production (S1) indicator is calculated as follows. The total supply of electricity is given by: (note we only consider domestic production and no imports or exports)

\begin{equation}
S_{total} = S_{solar} + S_{CCGT} + S_{wind} + S_{nuclear} + S_{hydro} + S_{hydrop} + S_{waste} + S_{ROR}
\end{equation}

The renewable supply is given by:

\begin{equation}
S_{RES} = S_{solar} + S_{wind} S_{hydro} + S_{hydrop} S_{ROR}
\end{equation}

The indicator is then normalised using:
\begin{equation}
S1 = S_{RES}  / S_{total}
\end{equation}

The electricity prices (S2) indicator is calculated based on the average electric price of the previous year. It is then normalised using an expected maximum electricity price ($P_{elec, max}$). The normalisation equation is given by:

\begin{equation}
S2 = \frac{P_{elec}}{P_{elec, max}}
\end{equation}

$P_{elec, max}$ is selected to be equal to 150 but can be tuned depending on the outcome of simulation such that the values of S2 are always between 0 and 1.

The renewable energy investment level (S3) indicator is calculated using the investment performed by all the investors in solar, wind and CCGT assets.

\begin{equation}
I_{total} = I_{wind} + I_{solar} + I_{CCGT}
\end{equation}

\begin{equation}
I_{RES} = I_{wind} + I_{solar}
\end{equation}

The indicator is normalised using:

\begin{equation}
S3 = I_{RES} / I_{total}
\end{equation}

The domestic level emissions (S4) is calculated based on the CCGT emissions. The normalisation of this indicator is once again done using an assumed maximum for the emissions which is given as five times the emissions for year 1 of the simulation. This is an arbitrary value that can be tuned to make sure that the indicator is always between 0 and 1.

\begin{equation}
S4 = S_{CCGT} / S_{CCGT, max}
\end{equation}

The imported emissions (S5) indicator is calculated based on the imports and the policy mixes of the countries from which Switzerland imports. The policy mixes are scenarios that are obtained from technical report and goals for the different countries. For each country, the percentage of coal and gas production is considered to calculated the imported emissions.

\begin{subequations}
\begin{align}
        E_{FR} & = (S_{FR, NTC} + S_{LTC}) \cdot (M_{FR, CCGT} \cdot E_{CCGT} + M_{FR, coal} \cdot E_{coal}) \\
         E_{DE} & = Ss_{DE, NTC} \cdot (M_{DE, CCGT} \cdot E_{CCGT} + M_{DE, coal} \cdot E_{coal}) \\
         E_{IT} & = S_{IT, NTC} \cdot (M_{IT, CCGT} \cdot E_{CCGT} + M_{IT, coal} \cdot E_{coal})
         \end{align}
\end{subequations}

where $E$ are the emissions per type of technology and where $M$ is the share of the mix for a specific technology in the country.

To normalise this indicator, we once again select a maximum amount of emissions. This is calculated as being 5\% higher than the initial imported emissions in year 1 of all these countries. Considering the emissions should decrease in the scenarios, this means that the indicator should remain within the $[0, 1]$ interval.

\begin{equation}
S5 = \frac{ E_{FR} + E_{DE} + E_{IT} }{ E_{total}}
\end{equation}

where $E_{total} = E_{FR, init} + E_{DE, init} + E_{IT, init}$


The policy core issues are given as the economy (PC1) and the environment (PC2). They are calculated using weighted averages of a number of secondary indicators. The equations used can be tuned but the ones implemented are given below:

\begin{equation}
PC1 = \frac{3}{4} \cdot S2 + \frac{1}{4} \cdot S3
\end{equation}

\begin{equation}
PC2 = \frac{1}{4} \cdot S1 + \frac{1}{4} \cdot S3 + \frac{1}{4} \cdot S4 + \frac{1}{4} \cdot S5
\end{equation}
				   

%%%%%%%%%%%%%%%% end of the subsection