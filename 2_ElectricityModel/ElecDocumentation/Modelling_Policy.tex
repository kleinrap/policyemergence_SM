\section{The policy process model}
\label{sec:CodeDocPolicy}

The policy emergence model uses concepts taken from the policy process theories as mentioned in the introduction. It follows work performed in \cite{klein2017emergence} and to be presented in forthcoming papers. This model has also been presented at a number of conference with the goal of obtaining feedback. This includes the International System Dynamics conference, the Social Simulation Conference, the International Conference on Energy Research and Social Science and the International Conference on Public Policy. The model is presented here using the ODD framework \citep{grimm2010odd}.

%%%%%%%%%%%%
\subsection{Purpose of the model}
\label{sec:purpose}

The purpose of the model is to simulate the policy process according to the Advocacy Coalition Framework (ACF) \citep{sabatier2007ACF}. By this, it is meant that the simulation should accommodate agents from a policy subsystem that can interact with one another based on their perception of the policy context - an electricity model in this case - and their respective interests. It should then enable these agents to decide whether to implement a policy instrument and if so, which one and at what time.

%%%%%%%%%%%% end of subsection

%%%%%%%%%%%%
\subsection{Entities, state variables and scales}
\label{ssec:entities}

The policy process simulation takes place at the policy subsystem level \citep{sabatier2007ACF}. The subsystem is selected based on the policy context of interest, represented here as the Swiss electricity market. This allows for the selection of the agents and the creation of specific structures within the model such as the agents' belief system.

Four different types of agents, in two categories, populate the model. The truth agent and the electorate are part of the passive agent family. The {\bfseries truth agent} passes information from the policy context onto the policy subsystem agents. This role has no equivalent in the real world, it is purely computational. The role of the {\bfseries electorate} is to influence the goals of the policy makers. Each electorate represents a political affiliation. They help shape the political field depending on their affiliation and goals \citep{laver2011party}. The model accommodates one electorate agent per political affiliation with a certain percentage of representativeness, corresponding to the amount of political support per affiliation.

The policy entrepreneurs and policy makers are part of the active agent category. Every active agent is a  {\bfseries policy entrepreneur}. This grants them the right to advocate for their interests. Some agents are also {\bfseries policy makers}. This grants them, additional decision making powers at a key step within the policy making process. They help select the agenda and they select the policy to be implemented.

All active agents have a number of attributes: a belief system composed of a {\bfseries problem tree}, {\bfseries resources}, and a {\bfseries policy} and {\bfseries affiliation network}. The problem tree is a three-tiered hierarchy composed of problems from the policy context following the ACF belief system \citep{sabatier1987knowledge}. The highest tier is composed of deep core beliefs which are normative values, the second tier is composed of policy core problems directly related to the policy context main problems while the lowest tier is composed of secondary problems related to details within the policy context. For each problem, the agents have a goal, a belief and, as a result of the difference between their goal and belief, a preference. This preference helps them select a specific problem of importance such that they can focus their limited attention on it. Finally, the problems are connected vertically with one another using causal relations. For example, more thermal power production can be perceived by the agents as having a negative impact on the investments into renewable energy. Overall, the problem tree provides a simplified representation of the policy context and its mechanisms within the mind of the agents.

Each agent has resources reflecting not only their financial resources but also the political resources \citep{nohrstedt2010logic}. These are used to interact with other agents.

Finally, all agents are connected through a policy network and an affiliation network. The policy network defines whether agents know each other and how much they trust one another. The affiliation network helps define the relations between the different political affiliations. This has an impact on the agents they talk to in the policy process.

Within the policy process, the agents can assemble into like-minded {\bfseries coalitions}. These coalitions are used by the agents to pull resources together to be more effective in their interactions with other agents. Such coalitions are created early in the policy process and remain stable throughout the process \citep{weible2018advocacy}. They are created with agents sharing similar policy core goals and beliefs. For example, in the present case two main coalitions will be formed: one focused on the environment and one focused on the economy \citep{markard2016socio}.

%%%%%%%%%%%% end of subsection

%%%%%%%%%%%%
\subsection{Process overview and scheduling}
\label{ssec:process}

The policy process considered is a two step process made of the {\bfseries agenda setting} and the {\bfseries policy formulation} step. This process is in part based on the theory of the policy cycle \citep{simmons1974policy}. Note that in the full hybrid simulation, the process is complemented by a simulation of the policy context, effectively adding one step to the policy process.

Before the start of the policy process, the agents are made aware of developments in the policy context. The indicators from the policy context simulation are calculated and fed to the truth agent which collects them unchanged. Then, these are passed on onto the active actors.

Once informed, agents select a problem that they consider to be most important in furthering their interests. These are the problem or policy they will advocate for throughout the entire process due to their limited attention span \citep{baumgartner2014punctuated}. During the agenda setting step, a policy core problem is selected. For the policy formulation step, a secondary problem and a policy instrument are selected.

In the agenda setting step, the agents interact with one another on their goals, beliefs, and understanding of the policy context (causal relations). The aim of these interactions is to align other agents with their own interests. Once they have completed their interactions, the agenda is selected. It is created if a majority of the agents agree on the same policy core problem. If no agenda is agreed upon, then the simulation skips the policy formulation and heads into the simulation of the policy context directly. If an agenda is created, the interactions between the agents continues on a narrower set of problems - secondary problems - in the policy formulation step.

The policy formulation step is slightly different. It ends with the selection, or lack thereof, of a policy instrument. This selection is performed by the policy makers only. If a majority of policy makers approves the same instrument, then it is selected and implemented within the policy context. If not, the status quo is maintained and the simulation continues undisturbed. Note that in this step as well, policy makers can be influenced by all other actors.

%%%%%%%%%%%% end of subsection

%%%%%%%%%%%%
\subsection{Design concepts}
\label{ssec:design}

%%
\paragraph{Basic principles}
%Which general concepts, theories, hypotheses, or modeling approaches are underlying the model?s design? Explain the relationship between these basic principles, the complexity expanded in this model, and the purpose of the study. How were they taken into account? Are they used at the level of submodels (e.g., deci- sions on land use, or foraging theory), or is their scope the system level (e.g., intermediate disturbance hypotheses)? Will the model provide insights about the basic principles themselves, i.e., their scope, their usefulness in real-world scenarios, validation, or modification (Grimm, 1999)? Does the model use new, or previously developed, theory for agent traits from which system dynamics emerge (e.g., ?individual-based theory? as described by Grimm and Railsback [2005; Grimm et al., 2005])?

The basic principles highlighted within this model is that through interaction, policy learning will be emulated. Policy learning is one of the pathways to policy change \citep{sabatier1988advocacy}. It comes as a result of the agents' changes in their belief system which is, in turn, a result of their reaction to the evolution of the policy context and their interactions with one another.

%%
\paragraph{Emergence}
%What key results or outputs of the model are modeled as emerging from the adaptive traits, or behaviors, of individuals? In other words, what model results are expected to vary in complex and perhaps unpredictable ways when particular characteristics of individuals or their environment change? Are there other results that are more tightly imposed by model rules and hence less dependent on what individuals do, and hence ?built in? rather than emergent results?

There are a number of emergent behaviours present in the policy emergence model. The main emergence behaviour is related to policy learning. Just as policy learning is an emergent phenomenon in real policy subsystems, it is also so within the simulation. It results from the interactions of the agents with one another but also from the reaction of the agents to the policy context.

Coalitions are another emergent phenomenon. Coalitions are created by agents with same-minded policy core interests. They are created to speed up the policy learning in the direction of the coalition's interests. These coalitions are expected to remain stable throughout the simulations considering the low speed at which policy core problems change. Coalitions can lead to a significant drive of the policy learning process, and lead to policy change.

Finally, the agenda and the policy instrument implementation can also be seen as emergent phenomena. They are the results of agents converging over and over in their beliefs on certain problems and policies. This convergence is the result of policy learning, the interactions between agents, the influence of the coalitions, and what is going on in the policy context.

%%
\paragraph{Adaptation}
%What adaptive traits do the individuals have? What rules do they have for making decisions or changing behavior in response to changes in themselves or their environment? Do these traits explicitly seek to increase some measure of individual success regarding its objectives (e.g., ?move to the cell providing fastest growth rate?, where growth is assumed to be an indicator of success; see the next concept)? Or do they instead simply cause individuals to reproduce observed behaviors (e.g., ?go uphill 70\% of the time?) that are implicitly assumed to indirectly convey success or fitness?

The agents have no strategy for say and therefore no adaptation possibilities. They follow rules which dictate that they can only select one problem at at time. They adapt their interests based on their understanding of the policy context, their goals, and the influence of other agents. Any change in their beliefs, goals or understanding will lead to a change in their preferences and therefore the interests they advocate for. This can effectively be seen as a change in their strategy as what they advocate will change over time.

%%
\paragraph{Objectives}
%If adaptive traits explicitly act to increase some measure of the individual?s success at meeting some objective, what exactly is that objective and how is it measured? When individuals make decisions by ranking alternatives, what criteria do they use? Some synonyms for ?objectives? are ?fitness? for organisms assumed to have adaptive traits evolved to provide reproductive success, ?utility? for economic reward in social models or simply ?success criteria? (note that the objective of such agents as members of a team, social insects, organs ? e.g., leaves ? of an organism, or cells in a tissue, may not refer to themselves but to the team, colony or organism of which they are a part).

The objectives of the agents are to bridge the gap between their goals and beliefs for all problems and above all else, their deep core problems. They do this principally through the implementation of policy instruments that will affect the policy context. This gap can also be influenced by other agents. Overall, and this relates to the core of the policy making process, and the agents are never able to reach their objectives fully. This can be due to unattainable objectives, the presence of unlimited and unexpected external events, a dynamic and unstable policy context or their flawed understanding of the policy context.

%%
\paragraph{Learning}
%Many individuals or agents (but also organizations and institutions) change their adaptive traits over time as a consequence of their experience? If so, how?

The agents have only one learning possibility: interactions. Every interaction they perform allow them a brief peak within the belief system of the agents they have interacted with. This allows them to be better informed on other agents and perform better informed interactions in the future. Agents do not have a memory and cannot inform their future decisions based on past interactions.

%%
\paragraph{Sensing}
%What internal and environmental state variables are individuals assumed to sense and consider in their decisions? What state variables of which other individuals and entities can an individual perceive; for example, signals that another individual may intentionally or unintentionally send? Sensing is often assumed to be local, but can happen through networks or can even be assumed to be global (e.g., a forager on one site sensing the resource lev- els of all other sites it could move to). If agents sense each other through social networks, is the structure of the network imposed or emergent? Are the mechanisms by which agents obtain information modeled explicitly, or are individuals simply assumed to know these variables?

All agents are provided with information on the policy context, through the truth agent. This information can be imperfect. The agents have virtually no way of establishing whether the information is correct or not beyond interacting with one another.

%%
\paragraph{Interaction}
%What kinds of interactions among agents are assumed? Are there direct interactions in which individuals encounter and affect others, or are interactions indirect, e.g., via competition for a mediating resource? If the interactions involve communication, how are such communications represented?

All interactions between the agents are explicit and relate to efforts that can be seen as lobbying, influencing or pressuring other agents on their belief system. Agents interact with one another to push their respective interests onto other agent's belief systems. The ultimate aim being to implement policy instruments they think are best.

%%
\paragraph{Stochasticity}
%What processes are modeled by assuming they are random or partly random? Is stochasticity used, for example, to reproduce variability in processes for which it is unimportant to model the actual causes of the variability? Is it used to cause model events or behaviors to occur with a specified frequency?

Stochasticity plays only a small role in a number of parts of the model. For example, agents are called upon in a random order when they perform interactions. Additionally, the knowledge they gain about other agent's belief system from their interactions is not exact. It is dependent on a small level of uncertainty. Finally, most of the inputs to the model, when it comes to the agent's belief systems, are introduced with a small dose of uncertainty.

%%
\paragraph{Collectives}
%Do the individuals form or belong to aggregations that affect, and are affected by, the individuals? Such collectives can be an important intermediate level of organization in an ABM; examples include social groups, fish schools and bird flocks, and human networks and organizations. How are collectives represented? Is a particular collective an emergent property of the individuals, such as a flock of birds that assembles as a result of individual behaviors, or is the collective simply a definition by the modeler, such as the set of individuals with certain properties, defined as a separate kind of entity with its own state variables and traits?

The agents can assemble into coalitions. The main effect of these coalitions is the ability to create what can be seen as "super-agents". They behave like agents with a lot more resources to push their respective interests forward and using what is effectively a greater policy network.

%%
\paragraph{Observation}
%What data are collected from the ABM for testing, understand- ing, and analyzing it, and how and when are they collected? Are all output data freely used, or are only certain data sampled and used, to imitate what can be observed in an empirical study (?Virtual Ecologist? approach; Zurell et al., 2010)?

A lot of data can be observed from the simulation. Amongst other things, there is the potential to observe all of the beliefs of all of the agents at all points in time throughout the simulation. However, this would lead to an enormous amount of data, and difficulty for analysis. Instead, the focus is placed on observing the agendas, the policy instruments selected and the different preferences for all agents. This allows the tracking of policy change. Then, depending on the focus of specific studies and the research questions selected, certain parts of the model can be observed such as the evolution of the network, the evolution of the coalitions or the influence of partial knowledge on the decision making of the agents. More details are provided on this later on.

%%%%%%%%%%%% end of subsection

%%%%%%%%%%%%
\subsection{Input data}
\label{ssec:inputData}
Both empirical data and modeller generated data can be used as inputs. This depends on the case being studied. In the present case where the model is coupled to an electricity model, the data used is empirical data obtained in other studies.

%%%%%%%%%%%% end of subsection

%%%%%%%%%%%%
\subsection{Submodels}
\label{ssec:submodels}

Two submodels are detailed: the influence of the electorate on policy makers and the active agent interactions.


%%
\paragraph{Electorate influence}

At the beginning of the policy process, the electorate influences the goals of the policy makers. Their goals follow those of their respective electorate in an effort to satisfy their electorate and remain in power \citep{laver2011party}. This happens throughout the simulation and is one of multiple ways that the goals of the policy makers evolve over time.

%%
\paragraph{Agent interactions}

The active agents can interact with one another. Such interactions can be performed on all other active agents and their belief system. An agent decides on specific actions based on the expected impact of the action. For this, the agent will be grading all actions possible. This grading takes into account the conflict level that s/he has with the other agent based on his/her understanding of the other agent's belief system and their mutual trust. It also accounts for the type of agent being influenced. Policy makers are preferred as they have more decision making power for example. The expected best action is selected. When an action has been selected, it is implemented by the agent.

%%
\paragraph{Policy network maintenance}

To perform actions and interactions, agents must have a robust policy network. To keep that network robust, they need to maintain it. They do so by spending resources to interact with other agents and maintain their level of trust with them. Resources spent on network maintenance are resources that cannot be spent on problems actions and interactions.

%%
\paragraph{Coalition creation}

Advocacy coalitions are created based on similarity of policy core problems goals. Agents that have similar goals will converge into a coalition.

%%
\paragraph{Coalition actions and interactions}

The coalitions can be seen as super-agents. They have a policy network and resources. They can also perform similar actions to the agents. The main difference is that the coalition actions are decided by the leader of the coalition. Furthermore, the actions can be performed on the members of the coalition themselves as an exercise of coalition strengthening, or they can be performed on agents outside the coalition as a way to push forward their interests.

%%%%%%%%%%%% end of subsection