%%%%%%%%%%%%%%%%%%%%%%%%%%%%%%%%%%%
\section{The steps for model integration}
\label{sec:steps}

This section presents the steps that are needed to connect a policy context model, in this case the predation model, to the policy process model.

\begin{enumerate}
\item Before any coding, define what the belief tree and the policy instruments will be for the predation model.
\item Copy the policy emergence model files into the same folder.
\item In \texttt{runbatch.py}, replace the policy context items by the predation model.
\item In \texttt{runbatch.py}, make sure to initialise the predation model appropriately.
\item Change the \texttt{input goalProfiles} files to have the appropriate belief tree structure of the predation model.
\item In \texttt{model module interface.py}, construct the belief tree and the policy instrument array.
\item Make sure that the step function in the \texttt{model predation.py} returns the KPIs that will fit in the belief system in the order DC, PC and S. If no DC is considered, then include one value of 0 at least. All KPIs need to be normalised.
\item Modify the step function of the \texttt{model predation.py} to include a policy implemented.
\item Introduce the changes that a policy implemented would have on the model in \texttt{model predation.py}.
\end{enumerate}

%%%%%%%%%%%%%%%%%%%%%%%%%%%%%%%%%%% end of the section


%%%%%%%%%%%%%%%%%%%%%%%%%%%%%%%%%%%
\section{The steps for model simulation}
\label{sec:steps}

This section presents the steps that are needed to connect a policy context model, in this case the electricity model, to the policy process model.

\begin{enumerate}
\item For the policy process:
	\begin{enumerate}
	\item Define a set of hypotheses to be tested
	\item Define scenarios that will be needed to assess the hypotheses
	\item Choose the agent distribution based on the scenarios constructed
	\item Set the preferred states for the active agents and the electorate along with the causal beliefs to be used. This should all be based on the scenarios that have been constructed.
	\end{enumerate}

\item For the predation model:
	\begin{enumerate}
	\item Define the initial values for the main parameters
	\item Define the parameters that will be recorded
	\end{enumerate}
\item Save the right data from the model.
\end{enumerate}

%%%%%%%%%%%%%%%%%%%%%%%%%%%%%%%%%%% end of the section